\subsection*{Definición}
La \emph{Cinemática} estudia el movimiento de los cuerpos sin considerar las fuerzas que lo producen. En robótica de manipuladores, las dos ramas principales son:

\begin{enumerate}
	\item \textbf{Cinemática Directa:} dado un conjunto de ángulos articulares, se calcula la posición y orientación del efector final en el espacio cartesiano.
	\item \textbf{Cinemática Inversa:} dada una posición/dirección deseada del efector final, se resuelven los ángulos articulares necesarios.
\end{enumerate}

\subsection*{Notación Denavit–Hartenberg}
Para sistematizar la obtención de las ecuaciones de movimiento, se emplea la convención DH:
\[
{}^{i-1}T_i = 
\begin{bmatrix}
	\cos\theta_i & -\sin\theta_i & 0 & a_{i-1} \\
	\sin\theta_i\cos\alpha_{i-1} & \cos\theta_i\cos\alpha_{i-1} & -\sin\alpha_{i-1} & -\sin\alpha_{i-1}d_i \\
	\sin\theta_i\sin\alpha_{i-1} & \cos\theta_i\sin\alpha_{i-1} & \cos\alpha_{i-1} & \cos\alpha_{i-1}d_i \\
	0 & 0 & 0 & 1
\end{bmatrix}.
\]
