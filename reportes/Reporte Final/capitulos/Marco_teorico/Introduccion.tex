\chapter{Marco Teórico} 
\label{chap:marco_teorico}

Esta es una pequeña introducción que provee una visión general del Marco Teórico que sustenta el proyecto de robótica. 

Aquí deben definir qué es la \textit{Cinemática}, la \textit{Dinámica} y el uso de \textit{ROS} en robótica. Deben explicar brevemente cómo cada una de estas disciplinas contribuye al control y simulación de manipuladores. Cabe señalar que las ecuaciones principales pueden obtenerse del las diapositivas de PowerPoint o desde los códigos de MATLAB. El apartado de ROS debe completarse con una pequeña búsqueda en Internet para describir su arquitectura básica y casos de uso en simulación.
